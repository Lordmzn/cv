%\RequirePackage{scrlfile}

%\PreventPackageFromLoading{ucs,inputenc}

\documentclass[totpages,helvetica,openbib,italian,flagCMYK]{europecv}

%\makeatletter
%\renewcommand\ecv@https://www.overleaf.com/5588307fvxtvd#utf[1]{{\inputencoding{utf8}#1}}
%\makeatother

%\ResetPreventPackageFromLoading
%\usepackage[utf8]{inputenc}
%\usepackage[T1]{fontenc}
\usepackage[a4paper,top=1.27cm,left=1cm,right=1cm,bottom=2cm]{geometry}
\usepackage[italian]{babel}
\usepackage{graphicx} % Required to draw the logo
\usepackage{bibentry} % to have entries of the bibliography in the cv
\usepackage{url}

% \renewcommand{\ttdefault}{phv} % Uses Helvetica instead of fixed
% width font

% Define personal data
\ecvname{Mason Emanuele}
%\ecvaddress{via F. Abbiati, 2, 20148, Milano}
%\ecvtelephone[3468873597]{0331376632}
%\ecvemail{\url{emanuele.mason.it@gmail.com}}
\ecvnationality{Italiana}
\ecvdateofbirth{9 Luglio 1988}
\ecvbeforepicture{\raggedleft}
\ecvpicture[height=5cm]{enersem.JPG}
\ecvafterpicture{\ecvspace{-3.5cm}}
\ecvfootnote{Per informazioni aggiuntive: \url{http://europass.cedefop.eu.int}\\
\textcopyright Comunità Europea, 2003.}


\begin{document}
\selectlanguage{italian}
\begin{europecv}
\ecvpersonalinfo[20pt]

%\ecvitem{\large\textbf{Impiego ricercato/ Settore di
%competenza}}{\large\textbf{Facoltativo}}
		
\ecvsection{Esperienza professionale}

\ecvitem{\textbf{4 agosto 2021 - oggi}}{\large Responsabile sviluppo soluzioni digitali}
\ecvitem{Principali mansioni e responsabilità}{Mi occupo del coordinamento e della progettazione delle soluzioni informatiche di ENERSEM, a partire dall'EMS, una piattaforma in cloud per l'archiviazione di dati di monitoraggio energetico da impianti industriali. Supervisiono il lavoro dell'azienda informatica incaricata dello sviluppo, partecipo alla progettazione dell'architettura e delle funzionalità, programmo direttamente alcune funzionalità prototipali e mi occupo della gestione dei server cloud e dei servizi associati (mail, analytics, e simili). Mi occupo inoltre della supervisione di progetti di ricerca, sviluppando direttamente alcune attività, e contribuisco alla scrittura di progetti.}
\ecvitem[10pt]{Nome della società}{ENERSEM s.r.l.}

\ecvitem{\textbf{9 lug 2020 - 3 agosto 2021}}{\large Socio lavoratore}
\ecvitem{Principali mansioni e responsabilità}{Vedi descrizione alla voce precedente.}
\ecvitem[10pt]{Nome della società}{ENERSEM s.r.l.}

\ecvitem{\textbf{6 mar 2018 - 8 lug 2020}}{\large Consulente esterno di modellistica, programmazione e architettura software per ENERSEM s.r.l.}
\ecvitem{Principali mansioni e responsabilità}{Come consulente esterno, ho contribuito agli sviluppi di modelli energetici, sia di impianti produttivi che di uffici e terziario. Mi sono occupato nello specifico di analisi di big data e di machine learning per ottenere nuove informazioni e sintetizzare suggerimenti per migliorare l'uso dell'energia, la sua efficienza ed efficacia. Mi sono occupato anche di progettazione di sistemi di controllo, in particolare basati sull'approccio \textit{model predictive control}.}
\ecvitem{}{Infine, ho supervisionato e lavorato ai progetti di ricerca finanziati di ENERSEM: dal 2018 a dicembre 2019 ho lavorato al progetto TEPORE su algoritmi MPC per termostati smart e sull'analisi dei dati di occupazione degli ambienti con tecniche di machine learning quali \textit{Support Vector Machines} e altri algoritmi non supervisionati. Dal 2020, ho lavorato al progetto LombHe@t che si occupa di algoritmi avanzati per la gestione dei generatori domestici di calore e per la gestione di reti teleriscaldamento.}
\ecvitem[10pt]{Nome e indirizzo del datore di lavoro}{ENERSEM s.r.l.}

\ecvitem{\textbf{1 nov 2017 - 5 mar 2018}}{\large Collaboratore a supporto della ricerca}
\ecvitem{Principali mansioni e responsabilità}{Ho sviluppato e applicato un algoritmo per la modellizzazione della gestione storica di un bacino idroelettrico alpino (Cancano, nord Italia) in periodi diversi, focalizzandomi sulle evoluzione del mercato idroelettrico, per quantificare la risposta adattiva della gestione del sistema. Ho utilizzato algoritmi di Inverse Reinforcement Learning e un modello concettuale a parametri concentrati dell'impianto idroelettrico.}
\ecvitem[10pt]{Nome e indirizzo del datore di lavoro}{Prof. Andrea Castelletti, Politecnico di Milano}

\ecvitem{\textbf{1 feb 2017 - 15 apr 2017}}{\large Desk study River Restoration wiki}
\ecvitem{Principali mansioni e responsabilità}{Analisi dei benefici di progetti realizzati di ristorazione fluviale in Europa tramite la revisione di database esistenti, incluso le wiki dei progetti LIFE+ RESTORE e FP7-REFORM, per identificare le prove che i lavori di ristorazione fluviali sono efficaci nel contenere il rischio idrogeologico e migliorare lo status ecologico. In collaborazione con il CIRF, Centro Italiano per la Riqualificazione Fluviale.}
\ecvitem[10pt]{Nome e indirizzo del datore di lavoro}{Wetlands International - European Association}

\ecvitem{\textbf{1 mar 2015 - 31 giu 2017}}{\large Collaboratore di per attività di didattica integrativa}
\ecvitem{Principali mansioni e responsabilità}{
Corso di ``Modellistica e simulazione'' in italiano, A.A. 2015/2016 e 2016/2017: esercitazioni in Excel sulla simulazione di sistemi discreti, discretizzazione di sistemi continui, programmazione lineare e non lineare con il risolutore di Excel. 20 ore di laboratorio informatico.}
\ecvitem{}{Corso di ``Modellistica e simulazione'' A.A. 2015/2016: sistemi non lineari tempo continui, stabilità e risposta canonica, controllo e osservabilità dei sistemi lineari. 12 ore di esercitazione.}
\ecvitem{}{Corso di ``Analisi dei Sistemi'' in Italian A.Y. 2014/2015: sistemi lineari tempo discreto e continuo, sistemi non lineari, stabilità e risposte canoniche. 8 ore di esercitazioni pratiche e 4 ore di laboratorio informatico Matlab.}
\ecvitem[10pt]{Nome e indirizzo del datore di lavoro}{
Prof. Giorgio Guariso, Politecnico di Milano.

Prof. Alessandra Gragnani, Politecnico di Milano.
}
		
\ecvitem{\textbf{1 ott 2013 - 30 set 2014}}{\large Assegnista di ricerca presso Politecnico di Milano.}
\ecvitem{Principali mansioni e responsabilità}{Modellazione del bacino del fiume Rosso in Vietnam allo scopo di progettare le politiche di regolazione delle dighe idroelettriche presenti per la aumentarne la produzione e la protezione dalle inondazioni nella capitale Hanoi. Implementazione del modello in C++, esecuzione di ottimizzazione in locale e su cluster HPC con algoritmi genetici multiobiettivo e legge di controllo basata su reti neurali.}
\ecvitem[20pt]{Nome e indirizzo del datore di lavoro}{Prof. Rodolfo Soncini-Sessa, Politecnico di Milano.}

\ecvsection{Istruzione e formazione}
		
\ecvitem{\textbf{ott 2014 - 17 gen 2018}}{\large Dottorato di ricerca in Ingegneria dell'Informazione, \emph{cum laude}}
\ecvitem{Titolo}{Beyond full rationality: modelling tradeoff dynamics in multiobjective water management}
\ecvitem{Tema principale}{Il tema di ricerca che ho sviluppato riguarda i metodi per modellizzare matematicamente il comportamento dei gestori umani dei sistemi idrici in contesto multi obiettivo. Ho esteso i tradizionali approcci normativi basati sull'ipotesi di massimizzazione dell'utilità al contesto multi obiettivo, in cui coesistono molteplici comportamenti ottimali. Ho sviluppato un approccio non parametrico basato sulle tecniche di inverse reinforcement learning per identificare le molteplici funzioni utilità di un gestore, e un protocollo di negoziazione parametrico basato su agenti per identificare il compromesso tra i diversi obiettivi scelto dal gestore e come questo evolva nel tempo. Entrambi gli approcci sono stati applicati a casi studio artificiali e reali.}
\ecvitem{Competenze acquisite}{Gestione delle risorse naturali, con particolare attenzione alle risorse idriche; teoria del controllo; controllo di sistemi ibridi; algoritmi supervisionati di machine learning e algoritmi di reinforcement learning; sistemi multiagente; teoria dei giochi; aiuto alla decisione multi criterio; epistemologia della ricerca nelle scienze sociali; metodi sperimentali nell'ingegneria.}
\ecvitem[15pt]{Supervisori e dipartimento}{Prof Andrea Castelletti and Matteo Giuliani, PhD, Dipartimento di Elettronica, Informatica e Bioingegneria, Politecnico di Milano}

\ecvitem{\textbf{25 mag 2014 - 1 giu 2014}}{\large Spring school in \emph{Multi Criteria Decision Aiding}, Perugia 2014}
\ecvitem{Temi principali / competenze acquisite}{Aiuto alla decisione multi criterio: teoria del valore a molti attributi, metodi ELECTRE e PROMETHEE, processi gerarchici analitici.}
\ecvitem[15pt]{Nome e tipo d'istituzione}{Università degli studi di Perugia}

\ecvitem{\textbf{ott 2010 - mag 2013}}{\large Laurea magistrale (M.Sc) in Ingegneria per l'ambiente e il territorio, indirizzo di pianificazione e gestione delle risorse naturali}
\ecvitem{Votazione}{110/110 \emph{cum laude}}
\ecvitem{Temi principali / competenze acquisite}{Gestione delle risorse naturali, modelli numerici dei sistemi fluviali, sistemi non lineari e studio di biforcazioni e catastrofi, ottimizzazione multi obiettivo, valutazione d'impatto ambientale e pianificazione del territorio.}
\ecvitem[15pt]{Nome e tipo d'istituzione}{Politecnico di Milano}
		
\ecvitem{\textbf{ott 2012 - dic 2012}}{\large Visiting scholar presso PennState}
\ecvitem{Attività principali}{Visiting scholar presso il gruppo di ricerca di Hydroinformatics - Dipartimento di Ingegneria civile e ambientale, Pennsylvania State University, USA.}
\ecvitem[15pt]{Temi principali / competenze acquisite}{Uso di algoritmi genetici ed evolutivi multi obiettivo per problemi matematici complessi (MOEAframework e BorgMOEA).}
		
\ecvitem{\textbf{ott 2007 - set 2010}}{\large Laurea di primo livello in Ingegneria per il territorio e l'ambiente}
\ecvitem{Votazione}{102/110}
\ecvitem{Temi principali / competenze acquisite}{Materie di base per l'ingegneria: matematica, fisica, chimica, informatica. Elementi di economia, scienza delle costruzioni, geologia, idrologia, chimica dell'ambiente.}
\ecvitem[15pt]{Nome e tipo d'istituzione}{Politecnico di Milano}

\ecvitem{\textbf{set 2003 - lug 2007}}{\large liceo scientifico}
\ecvitem{Votazione}{92/100}
\ecvitem{Temi principali / competenze acquisite}{Letteratura italiana, latina e inglese, comprensione e traduzione di latino scritto, comprensione e conversazione in inglese, scrittura di saggi brevi e articoli di giornale; argomenti principali di storia, filosofia e storia dell'arte; conoscenza di base di matematica, fisica, chimica, biologia, geologia, astronomia, disegno tecnico e artistico.}
\ecvitem[20pt]{Nome e tipo d'istituzione}{Liceo Scientifico A. Tosi, Busto Arsizio}


\ecvsection{Votazione}
\bibliographystyle{plain}
\nobibliography{../publications}
\ecvitem{\textbf{Articoli a rivista}}{}
\ecvitem[10pt]{[A5]}{\bibentry{Mason2018sec}}
\ecvitem[10pt]{[A4]}{\bibentry{Bizzi2018more}} % M3O
\ecvitem[10pt]{[A3]}{\bibentry{Giuliani2016matlab}} % M3O
\ecvitem[10pt]{[A2]}{\bibentry{Amigoni2016IDEAS}}
\ecvitem[10pt]{[A1]}{\bibentry{Giuliani2015emodps}}
\ecvitem{\textbf{Atti di conferenza}}{}
\ecvitem[10pt]{[B11]}{\bibentry{Mason2017egu}}
\ecvitem[10pt]{[B10]}{\bibentry{Giuliani2016iEMSs}}
\ecvitem[10pt]{[B9]}{\bibentry{Amigoni2016AAMAS}}
\ecvitem[10pt]{[B8]}{\bibentry{Mason2016ewri}}
\ecvitem[10pt]{[B7]}{\bibentry{Mason2015agu}}
\ecvitem[10pt]{[B6]}{\bibentry{Giuliani2015ewri}}
\ecvitem[10pt]{[B5]}{\bibentry{Giuliani2014agu}}
\ecvitem[10pt]{[B4]}{\bibentry{Giuliani2014ifac}}
\ecvitem[10pt]{[B3]}{\bibentry{Giuliani2014egu}}
\ecvitem[10pt]{[B2]}{\bibentry{Micotti2014egu}}
\ecvitem[10pt]{[B1]}{\bibentry{Mason2013egu}}

\ecvsection{Premi}
\ecvitem{\textbf{apr 2016}}{IDEAS 2016 Best Paper Award -- [B10]} 
\ecvitem[20pt]{Da}{IDEAS Organizing committee and AAMAS Workshop Chairs}

\ecvitem{\textbf{lug 2012}}{Vincitore borsa ``Tesi all'estero''}
\ecvitem[20pt]{From}{Politecnico di Milano}

\ecvsection{Competenze}
\ecvmothertongue[15pt]{Italiano}
\ecvitem{\large Altre lingue}{}
\ecvlanguageheader{(*)}
\ecvlanguage{Inglese}{\ecvCOne}{\ecvCOne}{\ecvBTwo}{\ecvBTwo}{\ecvCOne}
\ecvlanguagefooter[10pt]{(*)}

\ecvitem[10pt]{\large Capacità e competenza di comunicazione}{
Documentazione e report -- Manuale del software DMMT software per le agenzie vietnamite nel contesto del progetto IMRR. Report scientifico finale sui risultati (Deliverable D7.1 del progetto IMRR)}
\ecvitem[10pt]{}{Workshop MODUS all'Università di Nizza Sophia Antipolis, presentando un lavoro dal titolo ``Concetti dietro ai modelli'', sviluppato durante il corso di ``Modellistica e Governo del Territorio'', gen 2012.}
\ecvitem[10pt]{}{Scrittura di articoli scientifici -- vedere sezione \emph{pubblicazioni}.}

\ecvitem{\large Capacità e competenze organizzative}{Gestione del conflitto -- Membro del ``Consiglio dei Garanti'' del Partito Democratico della provincia di Varese, ott 2013 - dic 2016.}
\ecvitem[10pt]{}{Decision making e gestione di team -- Membro della segretaria della sezione locale dei Giovani Democratici, Busto Arsizio, giu 2013 - gen 2015.}
	 	
\ecvitem[10pt]{\large Competenze informatiche}{}
\ecvitem{Cloud}{Javascript (React), Python (Flask), basi di Kubernetes, Google Cloud}
\ecvitem{Elaborazione dati}{Jupyter notebooks in Python con pandas, sklearn e statsmodels e visualizzazione (matplotlib, seaborn, bokeh).}
\ecvitem{Scientifiche}{Matlab, R, Jupyter notebooks}
\ecvitem{Scripting e programmazione}{C++, Java, Bash, Git}
\ecvitem{}{Libreria PyDMMT: https://github.com/Lordmzn/pydmmt}
\ecvitem{}{Modello MOGLE: https://github.com/Lordmzn/mogle}
\ecvitem{Web}{conoscenze di base di html, CSS, Javascript}
\ecvitem{}{progettazione di piccoli siti web: www.cnai.info, www.lordmzn.it}
\ecvitem{Strumenti di produttività personale}{Latex, suite Apple, MS Office, Adobe Photoshop, Adobe Illustrator} 
\ecvitem[10pt]{Sistemi operativi}{OSX; Linux Ubuntu; Windows.}

\ecvitem[10pt]{\large Licenza di guida}{Patente B.}
		
\ecvitem[10pt]{\large Interessi personali}{Pratico arti marziali dal 2010, con lunga esperienza in arti marziali cinesi tradizionali e più recentemente, in scherma storica europea. Sono cintura nera di Shaolin (Meihuaquan) e Tai Chi stile Yang presso la Federazione Europea Scuole Kung fu (FESK) e faccio parte del 1595 club di scherma storica. Sono anche un fan di Formula 1. Nel tempo libero, faccio escursioni in montagna e vado a concerti rock.}

		%\ecvsection{Annexes}
		%\ecvitem{}{1. Brief report on the activities performed abroad
		%for my thesis project; Politecnico di Milano scholarship
		%a.a. 2011/12 for thesis abroad. }
\end{europecv}
\vspace{15pt}
\flushright
Milano, \today, \\
\vspace{10pt}
\underline{\hspace{250pt}}
	
\null\vspace*{\stretch{1}}
\flushright
\begin{tabular}{rp{250pt}}
 & \small Autorizzo il trattamento dei dati personali contenuti nel mio curriculum vitae in base all’art. 13 del D. Lgs. 196/2003 (disponibile presso
 \url{http://www.garanteprivacy.it/garante/doc.jsp?ID=1311248}) e all’art. 13 GDPR 679/16.
 \\
 \\
& \small Autorizzo la pubblicazione sul sito istituzione del Politecnico di Milano (sez. Amministrazione Trasparente)  in ottemperanza al D. Lgs n. 33 del 14 marzo 2013 (e s.m.i.).
 \\
 &  \\ 
 & Firma:\\ 
 & \underline{\hspace{250pt}} \\
\end{tabular} 

\end{document}